\documentclass{article}
\usepackage{graphicx}

\begin{document}
\title{Aaron McKay - Research and Background}
\author{Aaron McKay}
\maketitle

% Your content from first assignment
\section{About Me}
I am a 2nd year PhD student at UCCS.  I haven't yet published a paper but I'm working on it.  My area of research I think is one of the most challenging in all of cybersecurity, Cyber Risk Quantification.  Its taken me some time to figure out how this problem is best approached and I think the answer lies in a deep classification and analysis of the interactions of Attackers and Defenders. There may be several derivations from this interaction, but since my attention is on cyber risk quantification, my focus is on "Susceptibility". 

\begin{figure}[h]
    \centering
    \includegraphics[width=0.5\textwidth]{images/Oh_Yeah.jpg}
    \caption{Oh Yeah}
    \label{fig:oh-yeah}
\end{figure}

\section{Research Code Experience}
\subsection{Repository Used and Implementation Decisions}
I explored the Netflix-Skunkworks RiskQuant repository 
(\url{https://github.com/Netflix-Skunkworks/riskquant}). While attempting to use 
the full repository, I encountered compatibility issues with numpy<1.19.0,>=1.16.0 
when trying to install with Python 3.12. Rather than downgrade Python, I decided to 
implement a simplified version focusing on the core risk quantification concepts.

The simplified implementation used Python's built-in libraries:
\begin{itemize}
    \item random: for Monte Carlo simulation
    \item statistics: for statistical analysis
    \item Python classes: to organize the risk model structure
\end{itemize}

\subsection{Testing Experience and Implementation Details}
I created a SimpleRiskModel class that implements Monte Carlo simulation to estimate 
potential losses from cyber incidents. The model was designed with these key components:
\begin{itemize}
    \item Class initialization with risk parameters (min loss, max loss, probability)
    \item Simulation method running 10,000 iterations
    \item Analysis method calculating key statistics
    \item Output formatting for clear result presentation
\end{itemize}

The testing process involved:
\begin{enumerate}
    \item Setting up a data breach scenario with realistic parameters
    \item Running multiple simulations to verify consistency
    \item Analyzing the distribution of results
    \item Validating that the average loss aligned with theoretical expectations
\end{enumerate}

Key model parameters were chosen based on typical cyber incident statistics:
\begin{itemize}
    \item Minimum loss: \$100,000 (typical small breach cost)
    \item Maximum loss: \$1,000,000 (potential major incident)
    \item Annual probability: 10\% (industry average for similar incidents)
    \item Iterations: 10,000 (for statistical significance)
\end{itemize}

\subsection{Output Examples}
\begin{verbatim}
Risk Analysis for Data Breach Scenario:
Average Annual Loss: $52,624.65
Maximum Potential Loss: $999,737.69
90th Percentile Loss: $0.00
\end{verbatim}

\subsection{Analysis}
The results show an average annual loss of about \$52,600, which aligns with the 
10\% probability of an incident occurring. The maximum potential loss approaches 
the upper limit of \$1,000,000, demonstrating the model's ability to capture 
worst-case scenarios. The 90th percentile being \$0 indicates that in most 
simulation runs (over 90\%), no loss occurred, which is consistent with the 
low probability of occurrence.
\section{Questions and Answers}
% This section will be used later for Q&A

\end{document}

