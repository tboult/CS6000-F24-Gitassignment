\documentclass{article}
\usepackage{graphicx} % Required for inserting images
\title{Week 1 Journal}
\author{Abdul Shakur Illiyasu}
\date{\today}

\begin{document}
\maketitle

\begin{figure}[ht]
    \centering
    \includegraphics[width=0.35\textwidth]{images/abdul.jpg}
    \caption{Abdul Shakur Illiyasu}
    \label{fig: abdul.jpg}
\end{figure}

\section{Goals and What I Hope to Learn}

My goal for CS6000 is to finish the semester as a better research student. I aim to strengthen my problem-solving and logical reasoning skills, particularly in breaking down complex problems. This course will prepare me for future research and improve my writing and presentation abilities.

As someone who transitioned directly from an undergraduate to a PhD program without research experience, I hope to learn the fundamentals of conducting good research, publishing quality papers, and critically evaluating research papers.

\subsection{Degree and Research Project}
I am pursuing a PhD in cybersecurity and will be conducting research under Dr. Change on the Mutual Randomness Project. This project involves image and depth camera sensing and processing, to extract randomness, information, and entropy. Like CloudFlare’s lava lamp-based entropy, this project focuses on the mutual randomness observed by two entities. It examines the relative positions, movements, and perspectives between these entities, which are expected to be highly correlated. The randomness and entropy generated will secure communication between the entities, equipped with cameras and mobility capabilities.

\subsection {Something Personal}
One of my core values is to help others without expecting anything in return. This mindset ensures that I avoid disappointment when someone is unable to reciprocate when I need help. By not expecting anything in return, I protect myself from potential disappointment, and this approach greatly influences how I manage my work and relationships.

\section{Output of a python script for a random number generator.}
Generated 1000 random numbers.
Mean: 50.5
Standard Deviation: 29.1
\begin{figure}[ht]
    \centering
    \includegraphics[width=0.35\textwidth]{images/rannumgen.jpg}
    \caption{Histogram of generated random numbers}
    \label{fig: rannumgen.jpg}
\end{figure}


\section*{Questions: }

Q1. Your topic seems very interesting. How do you extract randomness from a camera?

Answer1: Extracting randomness from a camera involves capturing dynamic or unpredictable elements in the visual data, such as changes in light, movement, or noise. In my project, its the synchronized distance between the two vehicles and other point data like linear and angular velocity/acceleration. This randomness can be harnessed to generate entropy, which is useful for applications like cryptography or secure communication.

Q2: How have your core values positively influenced you this semester? D.C.

Answer2: It has allowed me to build strong, genuine connections with peers and colleagues, fostering a sense of trust and collaboration. By not expecting reciprocation, I’ve been able to focus purely on the act of helping, which has kept me motivated and prevented feelings of frustration when others may not be able to assist me in return. This mindset has also helped me maintain a positive attitude in my research and academic endeavors, emphasizing the joy of shared knowledge and support.

\end{document}
